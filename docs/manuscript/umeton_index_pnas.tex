\documentclass[9pt,twocolumn,twoside]{pnas-new}

\templatetype{pnasresearcharticle}

\title{The Umeton index: A leadership-filtered modification of the h-index}

\author[a,1]{Renato Umeton}

\affil[a]{Office of Data Science, St.\ Jude Children's Research Hospital, Memphis, TN 38105}

\leadauthor{Umeton}

\significancestatement{The h-index treats all co-authorships equally, allowing researchers to inflate their metrics through middle-author positions on collaborative papers without ever leading independent research. We propose the Umeton index (U-index), which counts only papers where the researcher is first or last author---positions that in biomedical and life sciences denote primary research contribution or supervisory leadership. This binary filter provides a transparent, easily calculable metric that isolates research leadership impact from collaborative contributions, addressing a well-documented limitation of the h-index in fields with meaningful authorship position conventions.}

\authorcontributions{R.U.\ designed research, performed research, and wrote the paper.}

\authordeclaration{The author declares no competing interest.}

\correspondingauthor{\textsuperscript{1}To whom correspondence should be addressed. E-mail: renato.umeton@stjude.org}

\keywords{bibliometrics $|$ h-index $|$ authorship $|$ citation metrics $|$ research evaluation}

\begin{abstract}
I propose the index $U$, defined as the number of first-or-last-authored papers with citation number $\geq U$, as a useful index to characterize a researcher's leadership contributions to the scientific literature.
\end{abstract}

\dates{This manuscript was compiled on \today}

\begin{document}

\maketitle
\thispagestyle{firststyle}
\ifthenelse{\boolean{shortarticle}}{\ifthenelse{\boolean{singlecolumn}}{\abscontentformatted}{\abscontent}}{}

The h-index, introduced by Hirsch in 2005 \cite{hirsch2005}, has become the dominant metric for evaluating individual scientific output. Its elegant definition---a researcher has index $h$ if $h$ of their papers have each been cited at least $h$ times---balances productivity and impact in a single number. However, the h-index has a well-documented limitation \cite{bornmann2007,waltman2012}: it treats all co-authorships equally regardless of an author's role in the research.

In biomedical and life sciences, authorship position carries significant meaning \cite{wren2007}. First authorship typically indicates the person who conducted the primary research work, while last authorship denotes the senior scientist who supervised the project, secured funding, and provided intellectual leadership. Middle authorship positions reflect supporting contributions of varying magnitude. The h-index makes no distinction among these roles, granting equal credit to all authors.

This creates a problematic incentive structure. A researcher can inflate their h-index by accumulating middle-author positions on large consortium papers or collaborative works without ever leading independent research. This phenomenon has prompted calls for author-position-aware metrics \cite{post2018,ioannidis2016}.

Several approaches address this limitation. The Schreiber $h_m$-index uses fractional counting based on co-author number \cite{schreiber2008}. The Stanford/Ioannidis composite c-score incorporates citations to papers by author position into a multi-indicator framework \cite{ioannidis2016,ioannidis2019}. The h-leadership index assigns Gaussian-weighted credit favoring first and last positions \cite{jain2025}.

Here I propose a simpler approach: the Umeton index (U-index), which applies a binary filter to the h-index calculation, counting only papers where the researcher is first or last author.

\subsection*{Definition}

Let $P$ denote the complete set of publications by researcher $R$. The \textit{leadership subset} $L(R)$ is defined as
\begin{equation}
L(R) = \{p \in P : R \text{ is first or last author of } p\}.
\end{equation}
The Umeton index $U$ is then
\begin{equation}
U = \max\{u \in \mathbb{Z}^+ : |\{p \in L(R) : c(p) \geq u\}| \geq u\},
\end{equation}
where $c(p)$ denotes the citation count of paper $p$.

In plain language: \textit{a researcher has Umeton index $U$ if $U$ of their first-or-last-authored papers have each been cited at least $U$ times.}

Single-author papers qualify the author as both first and last. When multiple authors are designated as co-first or co-last (marked with asterisks or daggers), all such authors qualify. Corresponding authorship does not qualify, as practices vary substantially across fields.

\subsection*{Properties}

The U-index has several useful properties. First, it is bounded above by the h-index:
\begin{equation}
U \leq h,
\end{equation}
because $L(R) \subseteq P$. Second, $U$ is monotonically non-decreasing: adding a first-or-last-authored paper can only increase or maintain $U$. Third, $U$ is completely insensitive to middle-author positions---a researcher who publishes 100 papers as middle author on highly-cited consortium studies will see their h-index rise substantially while their U-index remains unchanged.

The U-index also exhibits characteristic career-stage patterns. Early-career researchers accumulate first-author papers through primary research. Mid-career researchers contribute both first-author (independent work) and last-author (supervised work) papers. Senior researchers predominantly add last-author papers as they supervise trainees. The pair $(h, U)$ thus provides richer information than either alone:

\begin{itemize}
\item $h \gg U$: Impact built primarily through collaborative contributions
\item $h \approx U$: Impact concentrated in leadership positions
\end{itemize}

\subsection*{Comparison with Related Metrics}

The U-index differs from existing approaches in its deliberate simplicity. The Stanford c-score \cite{ioannidis2016} tracks citation statistics across author position categories and combines them into a composite indicator via log transformation. The c-index family \cite{post2018} characterizes the h-core by author position, including second and second-to-last positions. The h-leadership index \cite{jain2025} assigns continuous weights via a Gaussian curve.

The U-index takes the limiting case: first/last positions receive weight 1, all other positions receive weight 0. This binary approach sacrifices granularity for transparency. A paper either counts or it does not, eliminating ambiguity and making the metric trivially verifiable from any bibliometric database that records author order.

\subsection*{Limitations}

The U-index assumes that first and last author positions carry meaning. This assumption holds in biomedical sciences, clinical medicine, and experimental sciences, but fails in fields with different conventions. In mathematics, theoretical physics, and economics, alphabetical author ordering is common. In large particle physics or astronomy collaborations, author lists follow consortium conventions. In such fields, the U-index is not meaningful.

Additionally, in hyperauthorship scenarios with hundreds of authors, distinguishing meaningful first/last authorship becomes problematic. The U-index may inadvertently credit authors at list boundaries due to alphabetical ordering rather than scientific contribution.

\subsection*{Practical Calculation}

The U-index can be calculated as follows:
\begin{enumerate}
\item Retrieve all publications for researcher $R$
\item Filter to papers where $R$ is first or last author
\item Sort by citation count (descending)
\item Find the largest $u$ where paper $u$ has at least $u$ citations
\end{enumerate}

This procedure is implementable in Scopus, Web of Science, PubMed (with OpenAlex for citations), and Dimensions. An open-source implementation is available at \url{https://github.com/renato-umeton/u-index}.

\subsection*{Discussion}

I do not propose the U-index as a replacement for the h-index. Rather, it serves as a complement that isolates leadership contributions. Consider two researchers with identical h-indices of 30: Researcher A has $U = 25$ (most h-core papers are first/last authored), while Researcher B has $U = 8$ (most are middle-authored). These profiles differ markedly in ways the h-index obscures.

Neither profile is inherently superior---science requires both leaders and collaborators---but the distinction matters for evaluative contexts such as assessing readiness for independent positions or research group leadership.

The metric's simplicity is intentional. Unlike weighted approaches that assign fractional credit through complex formulas, the U-index provides a transparent answer to a specific question: how many impactful works has this researcher led or supervised? The binary nature eliminates debates about appropriate weighting schemes.

I recommend reporting the U-index alongside the h-index. Additionally, users may decompose $U$ into first-author ($U_F$) and last-author ($U_L$) components for finer-grained analysis of whether a researcher's leadership impact comes primarily from conducting research or supervising it.

\subsection*{Conclusion}

The Umeton index offers a minimal, transparent modification to the h-index that isolates research leadership contributions. By counting only first-or-last-authored papers, it provides a metric resistant to inflation through middle authorship while remaining simple to calculate and interpret. For fields where authorship position denotes meaningful leadership roles, the U-index provides a useful complement to existing metrics.

\matmethods{No materials or methods are applicable to this theoretical proposal. The U-index can be calculated from publicly available bibliometric databases.}

\showmatmethods{}

\acknow{The author thanks the Office of Data Science at St.\ Jude Children's Research Hospital for support.}

\showacknow{}

\bibsplit[4]

\begin{thebibliography}{10}

\bibitem{hirsch2005}
Hirsch JE (2005) An index to quantify an individual's scientific research output.
\textit{Proc Natl Acad Sci USA} 102(46):16569--16572.

\bibitem{bornmann2007}
Bornmann L, Daniel H-D (2007) What do we know about the h index?
\textit{J Am Soc Inf Sci Technol} 58(9):1381--1385.

\bibitem{waltman2012}
Waltman L, van Eck NJ (2012) The inconsistency of the h-index.
\textit{J Am Soc Inf Sci Technol} 63(2):406--415.

\bibitem{wren2007}
Wren JD, et al.\ (2007) The write position: A survey of perceived contributions to papers based on byline position and number of authors.
\textit{EMBO Rep} 8(11):988--991.

\bibitem{post2018}
Post A, et al.\ (2018) c-index and subindices of the h-index: New variants to account for variations in author contribution.
\textit{Cureus} 10(5):e2629.

\bibitem{ioannidis2016}
Ioannidis JPA, Klavans R, Boyack KW (2016) Multiple citation indicators and their composite across scientific disciplines.
\textit{PLoS Biol} 14(7):e1002501.

\bibitem{ioannidis2019}
Ioannidis JPA, Baas J, Klavans R, Boyack KW (2019) A standardized citation metrics author database annotated for scientific field.
\textit{PLoS Biol} 17(8):e3000384.

\bibitem{schreiber2008}
Schreiber M (2008) To share the fame in a fair way, $h_m$ modifies $h$ for multi-authored manuscripts.
\textit{New J Phys} 10:040201.

\bibitem{jain2025}
Jain HA, Chandra R (2025) Research impact evaluation based on effective authorship contribution sensitivity: h-leadership index.
\textit{arXiv}:2503.18236.

\end{thebibliography}

\end{document}
